\documentclass[12pt]{article}
\usepackage[utf8]{inputenc}
\usepackage{graphicx}
\graphicspath{{images/}}
\begin{document}
\begin{center}
\huge\underline{Evolution of Modern Health Care System}
\end{center}
\begin{center}
 \includegraphics[scale=0.7]{nitlogo.png }
\end{center}
\vspace{1cm}
\begin{center}
   \emph{\large By}\\
\Large{ PIYUSH BISEN }\\
\large{Roll No- 21111036}\\
\large{Biomedical 1st Sem}\\
\end{center}
\\
\\
\\

\begin{aling}
\newpage
\LARGE{\underline{Introduction}}
\end{aling}
\begin{center}
 \includegraphics[scale=0.875]{1.jpeg}
\caption{MODERN HEALTH CARE}
\end{center}
%\centering
\\
\\
The 20th century witnessed many truly revolutionary advances in health care. Research into the causes of infectious diseases and the development of vaccines and pharmaceuticals quelled once-devastating illnesses such as polio and smallpox. The first successful organ transplant occurred in 1954, and now, thousands of transplants each year—more than 28,000 in 2007—are prolonging the lives of recipients (UNOS, 2008). Over the past decade alone, better understanding of the mechanisms that cause disease has improved the ability to prevent, diagnose, and treat common afflictions such as diabetes and heart disease. The innovation underlying such progress continues to advance and accelerate change, while many new technologies and medical interventions provide new options for care and treatment.
\\
\\
\\
\newpage
\begin{align}
\LARGE{\underline{Modern Health Care Performance}}
\end{align}
\begin{center}
    \includegraphics[scale=0.8]{2.jpeg}
\end{center}
\\
\\
Since 2000, more and more initiatives have been taken at the international and national levels in order to strengthen national health systems as the core components of the global health system. Having this scope in mind, it is essential to have a clear, and unrestricted, vision of national health systems that might generate further progress in global health. The elaboration and the selection of performance indicators are indeed both highly dependent on the conceptual framework adopted for the evaluation of the health systems performance. Like most social systems, health systems are complex adaptive systems where change does not necessarily follow rigid management models. In complex systems path dependency, emergent properties and other non-linear patterns are seen, which can lead to the development of inappropriate guidelines for developing responsive health systems. 
\\
\\
An increasing number of tools and guidelines are being published by international agencies and development partners to assist health system decision-makers to monitor and assess health systems strengthening including human resources development[29] using standard definitions, indicators and measures. In response to a series of papers published in 2012 by members of the World Health Organization's Task Force on Developing Health Systems Guidance, researchers from the Future Health Systems consortium argue that there is insufficient focus on the 'policy implementation gap'. Recognizing the diversity of stakeholders and complexity of health systems is crucial to ensure that evidence-based guidelines are tested with requisite humility and without a rigid adherence to models dominated by a limited number of disciplines. Healthcare services often implement Quality Improvement Initiatives to overcome this policy implementation gap. Although many of these initiatives deliver improved healthcare, a large proportion fail to be sustained. Numerous tools and frameworks have been created to respond to this challenge and increase improvement longevity. One tool highlighted the need for these tools to respond to user preferences and settings to optimize impact.
\\
\\
Health Policy and Systems Research (HPSR) is an emerging multidisciplinary field that challenges 'disciplinary capture' by dominant health research traditions, arguing that these traditions generate premature and inappropriately narrow definitions that impede rather than enhance health systems strengthening .HPSR focuses on low- and middle-income countries and draws on the relativist social science paradigm which recognises that all phenomena are constructed through human behaviour and interpretation. In using this approach, HPSR offers insight into health systems by generating a complex understanding of context in order to enhance health policy learning. HPSR calls for greater involvement of local actors, including policy makers, civil society and researchers, in decisions that are made around funding health policy research and health systems strengthening.
\\
\\
\\
\begin{align}
  \LARGE{\underline{The Coming Transformation}}  
\end{align}
\\
\\
The next decade will see major shifts in the design of health systems and health care, propelled by digital health, growing consumerism, mounting financial constraints, and accelerated by Covid-19.
\\
\\
The problems in our health care systems include subpar quality and patient safety, a misplaced focus on acute care rather than on prevention and population health, inadequate person centeredness, and unsustainable cost. The next decade will see considerable transformation in how health systems are designed, propelled by opportunities such as digital health, growing consumerism, and mounting financial constraints. The Covid-19 pandemic has also necessitated and accelerated significant transformations. The authors discuss gaps and barriers in the current design of health and health systems, and the needed escalation of transformation including transition from hospital-based systems to primary care, community, and social care–based systems. They also assess the future evolution of payment systems leading toward sustainable health, changes in provider roles, and the entrance of new nontraditional players.
\\
\\
Health care today is often characterized by mediocre quality, poor safety, and high costs.1 Though change usually comes slowly, the Covid-19 pandemic has demonstrated that it is possible to rapidly retool our systems if there is a strong enough stimulus.2 In this article, we examine the future of health care — how it should change over the next 10 years, and the key drivers to enable our systems to become learning health care systems, with improved outcomes.
\\
\\
\\
\begin{align}
 \LARGE{\underline{Recent Medical Innovations in Healthcare Industry}}
\end{align}
\\
\\
\begin{itemize}
    \item IoT
    \item EHRs (Electronic health records)
    \item Remote Care
    \item 3D Printing
    \item LASIK
    \item Retail Clinics
    \item Augmented Reality
    \item Precision Medicine
    \item Blockchain
\end{itemize}




\enddocument