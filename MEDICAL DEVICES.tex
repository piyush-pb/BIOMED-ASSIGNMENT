\documentclass[a4paper,12pt]{article}
\usepackage{graphicx}
\graphicspath{{images/}}
\usepackage[x11names]{xcolor}


%\pagecolor{Cornsilk3}
%inline
{\Huge \title{Basic Biomedical Assignment-I  \\ Medical Devices}}
\author{Piyush Bisen \\ Roll no- 21111036}


\usepackage[landscape]{geometry}
\begin{document}
\begin{figure}
\centering
\includegraphics[scale=0.7]{nitrr.jpeg}
\end{figure}



\maketitle

\clearpage

\tableofcontents
\clearpage
\large
\begin{figure}
\centering
\includegraphics[scale=1.03]{pen injector.jpeg}
\caption{PEN INJECTOR}
\end{figure}
\section{PEN INJECTOR :}

 \medspace
 
  An injector pen (also known as a medication pen) is a gadget that allows you to inject medication directly into your skin. Injector pens were first created in the 1980s with the goal of making injectable medication easier and more convenient to use, hence enhancing patient adherence. The main difference between injector pens and standard vial and syringe administration is that injector pens are easier to use for those with limited dexterity, poor vision, or who want portability to administer medicine on time. Injector pens also reduce the fear or aversion to self-injecting drugs, increasing the likelihood that the medication will be taken.
 \\
 \\

 Injector pens are widely used for drugs that must be injected repeatedly by a person over a short period of time, such as insulin and insulin analogues for diabetes therapy (called insulin pens). Many additional treatments, such as injectable therapies for diabetes, high cholesterol, migraine prophylaxis, and various monoclonal antibodies, are also accessible as injector pens. Injector pens have been demonstrated to be at least as successful as vial and syringe administration in studies, and polls have revealed that the great majority of individuals would prefer an injector pen to vial and syringe administration if one were available. Injector pens have outperformed vial and syringe delivery of insulin in type 2 diabetes after a gradual acceptance in the United States.
 \\
  \clearpage



\begin{figure}
\centering
\includegraphics[scale=1]{pill cam.jpeg}
\caption{CAMERA PILL }
\end{figure}
\section{CAMERA PILL :}

 \medspace

A pill camera is a piece of equipment used in the capsule endoscopy process. It was created in the late twentieth century and FDA-approved for use in 2001.
\\
\\
  The camera is roughly an inch long and half an inch wide, with rounded corners that make it look like a pill capsule (although slightly larger). It consists of a camera, flash, plastic capsule, and transmitter (often Bluetooth (TM) at the moment). It's small enough to be consumed whole.
\\
\\
  When a small intestine ailment is suspected, the pill camera is frequently used. An endoscope may normally inspect the upper digestive tract, but a colonoscopy is recommended for problems with the large intestine. Neither of these methods, however, allows for exploration of the small intestine. Furthermore, the pill camera is non-invasive. A pill camera, unlike endoscopy and colonoscopy, cannot be used to treat a disease. 
  \\
  \\
  Capsule endoscopy is considered to be a very safe method to determine an unknown cause of a gastrointestinal bleed.[11] The capsule is usually excreted with the feces within 24–48 hours.[11] There has been a report of retention of the capsule for almost four and a half years although the patient was asymptomatic.[11] However, the risk of bowel obstruction may be countered by abdominal X-ray to locate the device for removal by endoscopy or surgery.[11]
\clearpage
 
  
  
  


 \begin{figure}
\centering
\includegraphics[scale=0.7]{robotic.jpeg}
\caption{ROBOTIC SURGERY SYSTEM }
\end{figure}
  
\section{ROBOTIC SURGERY SYSTEM :}

 \medspace

 The use of a robotic surgical system to perform operations on patients is known as robotic surgery. It, like minimally invasive surgery, can be performed alone or in conjunction with traditional open surgical procedures, depending on the situation. The Da Vinci system is the most used robotic system over the world. It consists of three components - the surgeon’s console, a patient cart, and the vision cart. All of these elements work together to allow the surgeon to see what's going on and then to mimic the events in order to guide the instruments.
 \\
 \\
 The surgeon’s console is the place where the surgeon sits. This is the area where he sees what is happening and has master control of how the instruments need to move. He can view high-definition real-time 3D images at the console. The patient cart is kept next to the patient’s bed where he is being operated on. The patient cart holds the camera and the instruments that are required for the surgery. The vision cart is the third component that is in charge of enabling the communication to take place seamlessly between all the components. The components of different surgical systems may vary depending on the particular system.
 \\
 \\
The history of robotic surgery begins in the 1980s. The first robotic surgical system utilised for stereotaxic surgery was PUMA 560, which was introduced in 1985. PROBOT was then used in transurethral prostate surgery in 1988. ROBODOC, a femur cavity preparation tool, was created in 1992. Minimally invasive surgery was introduced in the 1990s with the introduction of laparoscopic surgery. By this time, three robotic surgical systems had been developed: the Da Vinci surgical system, Zeus, and AESOP. Computer Motion developed the AESOP and Zeus systems, while Intuitive Surgical Inc. created the Da Vinci surgical system.
 \\
 \\
 Computer Motion was acquired by Intuitive Surgical Inc. Within a few years, Zeus was discontinued. Today, the Da Vinci surgical system is the most used across the globe. There are presently four models of Da Vinci robotic surgical systems that are available; Da Vinci Si, Da Vinci X, Da Vinci Xi, and Da Vinci SP. These are designed for different types of robotic surgeries.
 \clearpage
 
 \begin{figure}
\centering
\includegraphics[scale=0.9]{TISSUE HOM.jpeg}
\caption{TISSUE HOMONGENIZER  }
\end{figure}
  \section{TISSUE HOMONGENIZER :}
 
 \medspace
  
Tissue homogenization is performed regularly in labs across the world for cell and tissue preparation. This process involves lysing the cells to release intracellular contents of interest, such as proteins and nuclear components.
  \\
  \\
  Homogenization to break tissues down into their constituent pieces is a common first step in the lab. Depending on the desired constituent parts, different techniques may be used. For more thorough homogenization, blending of the tissue is often done first and a disruptor is then used to break the tissue down even further. Homogenizers can have dedicated functions, such as a tissue-specific homogenizer, or more general functions. If only one type of homogenization is going to be done in the lab, a tissue homogenizer may be the most economical option, otherwise investing in a homogenizer with multiple applications may be something to consider.
  \\
  \\
  There are often many different names for the same piece of mechanical homogenizing equipment, including Cell Lysor, Disperser, High Shear Mixer, Homogenizer, Polytron, Rotor Stator Homogenizer, Sonicator or Tissue Tearor.
  \\
  
  Cell fractionation is done by homogenizer to release the organelles from cell. Whereas older technologies just focused on the disruption of the material, newer technologies also address quality or environmental aspects, such as cross-contamination, aerosols, risk of infection, or noise. Homogenization is a very common sample preparation step prior to the analysis of nucleic acids, proteins, cells, metabolism, pathogens, and many other targets.
  Mechanical Disruption-
  Involve the use of rotating blades to grind and disperse cells , most effective at homogenizing cell tissues, such as liver, can homogenize small volumes, up to 20 liters ,sample loss is minimal, sonication.
  Physical disruption used to lyse cells uses high frequency sound waves to lyse cells, bacteria, and other tissue types, time consuming and best used for small volumes (<100mL),manual grinding.
  \\
  \\
  Most widely used cell disruption technique, cells are lysed by the action of being forced through a small space, suitable for use with small volumes as well as cultured cells, liquid homogenization is the most commonly used homogenization technique. In the world of liquid homogenization, there are several types of homogenizers that are designed to complete the task: Potter-Elvehjem homogenizers, French Presses, and the Dounce Homogenizer.
  \\
  \\
  
In addition, we have extensive experience in the challenges that our customers face as they transition from concept, through to R and D, clinical trials, all important FDA approval and finally, to manufacturing.
\\
\\ 
\clearpage
\\

\begin{figure}
\centering
\includegraphics[scale=0.65]{Electron_Microscope.jpg}
\caption{ELECTRON MICROSCOPE  }
\end{figure}
\\
\section{ELECTRON MICROSCOPE :}


\medspace
 
An electron microscope is a microscope that illuminates with a beam of accelerated electrons. Electron microscopes offer a higher resolving power than light microscopes and may expose the structure of smaller objects since the wavelength of an electron is 100,000 times shorter than that of visible light photons. Most light microscopes are restricted by diffraction to roughly 200 nm resolution and usable magnifications below 2000, whereas a scanning transmission electron microscope has achieved more than 50 pm resolution in annular dark-field imaging mode[1] and magnifications up to about 10,000,000.
\\
\\
Shaped magnetic fields are used in electron microscopes to create electron optical lens systems that are similar to the glass lenses used in optical light microscopes.
\\
\\
Microorganisms, cells, big molecules, biopsy samples, metals, and crystals are among the biological and inorganic specimens that electron microscopes are used to analyse the ultrastructure of. Electron microscopes are frequently used in industry for quality control and failure analysis. Modern electron microscopes create electron micrographs by capturing images with sophisticated digital cameras and frame grabbers.
\\
\\
Ruska built the first electron microscope that exceeded the resolution attainable with an optical (light) microscope.[4] Four years later, in 1937, Siemens financed the work of Ernst Ruska and Bodo von Borries, and employed Helmut Ruska, Ernst's brother, to develop applications for the microscope, especially with biological specimens.[4][6] Also in 1937, Manfred von Ardenne pioneered the scanning electron microscope.Siemens produced a transmission electron microscope (TEM) in 1939.[clarification needed][10] Although current transmission electron microscopes are capable of two million-power magnification, as scientific instruments, they remain based upon Ruska's prototype.
 
\end{document}