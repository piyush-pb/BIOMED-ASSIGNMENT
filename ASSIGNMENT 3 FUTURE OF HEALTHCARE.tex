\documentclass[12pt]{article}
\usepackage[utf8]{inputenc}
\usepackage{graphicx}
\graphicspath{{images/}}
\begin{document}
\begin{center}
\huge\underline{FUTURE OF HEALTHCARE}
\end{center}
\begin{center}
 \includegraphics[scale=0.7]{nitlogo.png }
\end{center}
\vspace{1cm}
\begin{center}
   \emph{\large By}\\
\Large{ PIYUSH BISEN }\\
\large{Roll No- 21111036}\\
\large{Biomedical 1st Sem}\\
\end{center}
\begin{center}
\newpage
\huge{\underline{Future Of Healthcare}}
\end{center}
\\
Today, it's easier than ever for patients to access medical services outside of the traditional four walls of the medical establishment through the internet and mobile devices.
\\
\begin{center}
\includegraphics[scale=0.7]{3.jpeg}
\end{center}
\\
Telehealth has made it possible for patients to receive care without an in-person office visit. Remote patient monitoring is becoming more widely accepted. This now includes wearable technology with impressive capabilities, from remote monitoring of vitals to remote echocardiograms. If not for the pandemic, it would have taken the healthcare industry another decade to reach where it is today.
\\
\\
Companies have a massive opportunity to introduce the right tools to help both patients and healthcare professionals improve their patient's quality of life.
\\
\\
As we look to next year and beyond, I envision integrated data sharing, heightened transparency for patients and predictive analytics becoming paramount in healthcare.
\section{Data Sharing}
\\
\\
Patient data and information are not routinely shared across providers. This can cause avoidable challenges, frustrations, delays and potentially harmful outcomes for patients. Secure, HIPAA-compliant sharing of patient data will be one of the most important advancements in the coming decade.
\\
\\
Withholding patient data and information leads to the accrual of extraordinary amounts of unnecessary healthcare costs. This is due, in large part, to inessential or redundant medical labs and workups being done because providers do not have access to a patient's full medical history.
\\
\\
To help increase access to care and to improve the overall industry for both health systems and patients, using advanced technology to enhance data sharing is critical.
\section{Transparency Into the Medical Life Cycle}
\\
\\
The lack of transparency into specific services and costs has caused massive challenges and misunderstandings. It's all too common for patients to not fully know what medical service is being done, when it's being done and at what price point. Technology will help increase transparency into every step of the medical life cycle for patients.
\\
\\
Further, it’s important that this information is presented to patients in a way that is accessible to everyone, including those who are undereducated, underinsured and the elderly. A much-needed step to help increase access to quality healthcare for all, heightened transparency into the medical life cycle is key.
\section{Predictive Analytics}
\\
\\
We will continue to see more advanced technology such as artificial intelligence (AI) and machine learning (ML) leveraged for predictive analytics in healthcare. Valuable tools that are constantly undergoing innovation, AI and ML can help enable potentially life-saving predictions for patients. This includes:
\\
\\
• How a given patient is likely to react to certain medications or treatment protocols.
\\
• Which patients are likely to decompensate due to missed preventative treatments.
\\
• Which patients are likely to get readmitted to the hospital and for what reasons.
\\
• Early detection of infections, including detecting Covid-19, based on remote monitoring of vitals.
\\
For instance, Ochsner Health System uses an AI tool that enables doctors to predict when a patient is about to “code” — or suffer a cardiac or respiratory arrest.
\section{Technological Challenges}
\\
\\
The largest challenge the healthcare industry faces when it comes to adopting new technology is the initial error rate. Generally, new technological products require iteration before they’re sufficiently reliable. This iterative process can be painful, potentially resulting in inaccurate predictions and inappropriate recommendations.
\\
\\
To avoid this, technologists and clinicians must closely collaborate when rolling out new technology. Together, they will need to carefully test new tools and identify fail-safe methods until reliability is sufficiently achieved.
\section{Harnessing The Power Of Technology And Mobility}
\\
\\
It’s evident that now is the time for healthcare technology companies to instill true, lasting change that will improve the industry for patients and providers nationwide.
\\
This change will be made possible by the continued adoption of innovative technology combined with the rise of a mobile workforce. With certain medical employees not bound by a central physical location, more services can be given outside of the traditional medical establishment — including in a patient’s neighborhood or home.
\\
\\
Looking ahead, it’s essential that the healthcare industry remains focused on one common goal — ensuring that everyone, no matter personal circumstances, has access to high-quality and highly affordable care. Advanced technology, made even more powerful by increased mobility, will make this a reality.

\end{document}